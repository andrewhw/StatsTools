%% $Id: 014-Summary.tex,v 1.1 2005/04/05 15:07:55 andrew Exp 2

\documentclass{beamer}
\mode<presentation>{\usetheme{Warsaw}\usecolortheme{crane}}


\usepackage[latin1]{inputenc}
\usepackage{helvet}
\usepackage{graphicx}
\usepackage{xspace}
\usepackage{paralist}
\usepackage{amsmath,amssymb,bm}
\usepackage{amsthm}
\usepackage{multirow}
\usepackage{fancybox}
\usepackage{cancel}
\usepackage{alphalph}
\usepackage{listings}
\usepackage{xcolor}
\usepackage{pagecolor}
\usepackage{pdfcomment}
\usepackage{wasysym}
\usepackage{ulem}
\usepackage{svg}
\usepackage[weather]{ifsym}
\usepackage{colortbl}
\usepackage[linesnumbered,ruled,vlined]{algorithm2e}
\usepackage{listings}
\usepackage{hyperref}

\usepackage[T1]{fontenc}
\usepackage[default]{gillius}

\definecolor{grey}{rgb}{0.6,0.6,0.6}
\definecolor{burntumber}{rgb}{0.55, 0.2, 0.14} %0.8,0.7,0.0}
\definecolor{orchid}{rgb}{0.294, 0, 0.510} %0.8,0.7,0.0}	51,51,178


% set up symbol based footnotes
\makeatletter
\newcommand*{\fnsymbolsingle}[1]{%
\ensuremath{%
    \ifcase#1%
    \or *%
    \or \dagger
    \or \ddagger
    %\or \mathframetitle
    %\or \mathparagraph
    %\or \triangleup
    \or \triangleleft
    \or \bowtie
    \or \oplus
    %\or \maltese
    \else
    \@ctrerr
    \fi
}%
}
\newalphalph{\fnsymbolwrap}[wrap]{\fnsymbolsingle}{}
\makeatother
\renewcommand\thefootnote{\fnsymbolwrap{\value{footnote}}}



\newcommand{\eg}{\textit{e.g.},\xspace} % some Latin abreviations in italic
\newcommand{\ie}{\textit{i.e.};\xspace}
\newcommand{\aka}{\textit{a.k.a.}\xspace}
\newcommand{\etc}{\textit{etc}.\@\xspace}
\newcommand{\theref}{\ensuremath{\dot{.~.}~}}

\newcommand{\key}[1]{\textcolor{orchid}{{\bf #1}}}

\newcommand{\kq}[1]{\textcolor{burntumber}{$\mathcal{Q}$: \emph{#1}}}


%	----------------------------------------------------------

\title{Testing for Significant Differences Between Distributions}
\subtitle{A Crib Sheet}

\author{Dr.~Andrew Hamilton-Wright}
\institute{School of Computer Science\\
University of Guelph}

\date{\today}



\begin{document}

% Get rid of beamer's default navigation decoration
\setbeamertemplate{sidebar right}{}


\begin{frame}
    \titlepage
\end{frame}


\begin{frame}
\frametitle{The overall question}
If you have done an experiment and collected some sets of
measures from two or more experimental setups, a common
question is:
\begin{quote}
\kq{Is there a significant difference between the distributions represented by the points making up my data sets?}
\end{quote}

This set of slides gives you some advice on how to proceed to answer
this question.

More detailed answers can be found in excellent online sources.  My
two favourites are these -- they both give excellent explanations:

\hangindent=3em
\textcolor{orchid}{McDonald, J.H. (2014): \emph{Handbook of Biological Statistics} (3rd ed.).
		Sparky House Publishing, Baltimore, Maryland.
		\url{http://www.biostathandbook.com/index.html}}

\hangindent=3em
\textcolor{orchid}{NIST/SEMATECH (2013): \emph{e-Handbook of Statistical Methods}, \url{http://www.itl.nist.gov/div898/handbook/}}
\end{frame}


\begin{frame}
\frametitle{Preliminary Tools}
Some questions will depend on the answers to one or both of
these questions, as applied to each of your sets of points:
\begin{itemize}
\item \kq{Is this set of points normally distributed?}
	\pause
	\begin{itemize}
	\item \key{Shapiro-Wilk} (the most sensitive)%
			\footnote{\textcolor{grey}{Razali \& Wah (2011):
					Power comparisons of Shapiro-Wilk, Kolmogorov-Smirnov, Lilliefors and Anderson-Darling Tests
					\emph{Journal of  Statistical Modelling and Analytics} 2(1):21--33. ISBN:~978-967-363-157-5\\ \href{https://www.researchgate.net/profile/Bee-Yap/publication/267205556_Power_Comparisons_of_Shapiro-Wilk_Kolmogorov-Smirnov_Lilliefors_and_Anderson-Darling_Tests/links/5477245b0cf29afed61446e1/Power-Comparisons-of-Shapiro-Wilk-Kolmogorov-Smirnov-Lilliefors-and-Anderson-Darling-Tests.pdf}{Link via ResearchGate (Apr 2021)}}}
	\item \key{Anderson-Darling},
	\item \key{Kolmogorov-Smirnov}
	\end{itemize}
	\vspace{1em}\pause
\item \kq{Are the variances between two of my sets of points different?}
	\pause
	\begin{itemize}
	\item \key{$F$-test (of equality of variances)}
	\end{itemize}
\end{itemize}
\end{frame}

\begin{frame}
\frametitle{Statistics: \visible<6->{Independent two-sample $t$-test}}
\kq{Is there a significant difference between the distributions represented by the points making up my data sets?}\\
If all of the following are true:
\begin{itemize}
\item you have \key{exactly two} distributions of data (sets of points), \pause
\item and they are \key{normally distributed} \pause
\item and your observations are \key{independent} \pause
\item and your variances are equal (\aka \key{homoscedasticity}) \pause
\end{itemize}
then \pause
\begin{itemize}
\item you can use the \key{(independent two-sample) $t$-test}
to answer your question.
\end{itemize}
\end{frame}

\begin{frame}
\frametitle{Statistics: \visible<2->{ANOVA}}
\kq{Is there a significant difference between the distributions represented by the points making up my data sets?}\\
If all of the following are true:
\begin{itemize}
\item you have \key{\bf more than two} distributions of data (sets of points),
\item and they are \key{normally distributed} 
\item and your observations are \key{independent}
\item and your variances are equal (\aka \key{homoscedasticity}) 
\item and your question is ``\kq{do these samples come
		from distributions with the same mean}?''
\end{itemize}
then \pause
\begin{itemize}
\item you can use \key{ANOVA} (Analysis Of VARiance)
\end{itemize}
but \pause
\begin{itemize}
\item this only tells you that \emph{one or more} of the means of
	your distributions is different from the others, and doesn't tell
	you which one.
\item \pause You run a \key{post hoc test} to determine that.
\end{itemize}
\end{frame}

\begin{frame}
\frametitle{Statistics: \visible<5->{Paired sample $t$-test}}
\kq{Is there a significant difference between the distributions represented by the points making up my data sets?}\\
If all of the following are true:
\begin{itemize}
\item you collected your points in \key{paired observations}
from a set of subjects or entities, so that you have \key{two measures}
from the \key{same subject} under two conditions, and you want to know
whether the change in condition produced a change in measurement
between the \key{paired measures}
\item and \key{these differences} are \key{normally distributed} \pause
\item and your observations \key{between subjects} are \key{independent} \pause
\item and your question is ``\kq{do the differences described by
		these pairs have a non-zero mean}?'' (\ie ``\kq{Is there
			a difference between these paired values}?)
\end{itemize}
then \pause
\begin{itemize}
\item you can use the \key{(paired) $t$-test}
to answer your question.
\end{itemize}
\end{frame}


\begin{frame}
\frametitle{Statistics: Paired sample $t$-test}
Here is something that people frequently get mixed up for
the paired $t$-test \dots

\begin{block}{The Normality requirement only applies \key{to the differences}}
Note that only the \key{differences} calculated between the pairs
need to be normally distributed -- the distributions of the
unpaired points \emph{does not matter at all}.

This is because the paired $t$-test \key{converts the set of differences
into a single distribution}, and uses a $t$-test to determine if \key{this
distribution} is non-zero.
\end{block}
\end{frame}

\begin{frame}
\frametitle{Bonferroni correction}
If you have point sets
$\mathcal{A}$,
$\mathcal{B}$,
$\mathcal{C}$,
and want to make \key{multiple comparisons}, such as
$\mathcal{A}$:$\mathcal{B}$ and
$\mathcal{B}$:$\mathcal{C}$, then
use a correction such as the \key{Bonferroni correction}:
\begin{itemize}
\item used to correct the estimation done when multiple inferences
		are made on the same data (\aka the ``\key{multiple comparisons} problem)
\pause
\item built into most statistical packages, but math is trivial
\item we just:
	\begin{itemize}
	\item divide our $\alpha$ by the number of tests we are doing (making it lower), or equivalently
	\pause
	\item multiply our $p$ by the number of tests we are doing (making them higher).
	\end{itemize}
\end{itemize}
\pause
\begin{block}{Why do we do this?}
This decreases our likelihood of saying that any test has produced
		a statistically signficant result --- because we are attempting
		to control for the likelihood of finding such a result by
		random chance.
\end{block}
\end{frame}

\begin{frame}
\frametitle{Statistics: Non-parametric tests}
\kq{Is there a significant difference between the distributions represented by the points making up my data sets?}\\
\begin{itemize}
\item If your data are \key{NOT normally distributed}
	\begin{itemize}
	\item or your variance does not conform to your constraints
	\end{itemize}
\item but your observations are \key{independent}
\end{itemize}
then you can use:\pause
\begin{itemize}
	\item \key{Mann-Whitney-Wilcoxon} (MWW)\footnote{
			\visible<3->{This is also called the ``\key{Mann-Whitney $U$ test}'' or ``\key{Wilcoxon-rank-sum}'' ---
			but note that the ``\sout{\key{Wilcoxon signed-rank}}'' test is a different beastie (more on next slide).}}
		   	if there are \key{two} distributions
		(analogue of the $t$-test, but with no model), \visible<3->{or}
	\item \visible<4->{Kruskal-Wallis (KW) if there are \key{more than two}.}
\end{itemize}
\vspace{1em}
%\begin{block}{KW {\it post hoc} strategies}
\visible<5->{Note that KW test functions like a \key{{\bf one-way} ANOVA}, \pause
and to determine which pairs differ, you can use MWW \emph{without}
the \key{Bonferroni correction}.}
%\end{block}
\end{frame}


\begin{frame}
\frametitle{Non-independent tests}
\kq{Is there a significant difference between the distributions represented by the points making up my data sets?}\vspace{1em}

If your data are \key{NOT independent} then 
\begin{itemize}
\item if you have \key{two} distributions
		(and \theref would have considered \key{Mann-Whitney-Wilcoxon})
	%\footnote{(\aka \key{Wilcoxon rank sum}/\key{Mann-Whitney $U$ test})}
		then \pause
			use the \key{Wilcoxon signed-rank} test
		and \pause
\item if you have \key{more than two} distrubutions
		(and would have considered \key{Kruskal-Wallis}), then \pause
		use the \key{Friedman} test instead.
\end{itemize}
\end{frame}

\begin{frame}
\frametitle{Summary: Which test to use?}
\kq{Is there a significant difference between the distributions represented by the points making up my data sets?}\vspace{1em}

\begin{center}
\begin{tabular}{cccc@{}c}
{\it Independence}
		& {\it Normality}
			& {\it $N$ dist.}
				& $\rightarrow$ & {\it Use} \\
\hline
\hline
\multirow{4}{*}{Yes}
		&	\multirow{2}{*}{Yes}
			&	2
				&& \visible<2->{\key{$t$-test}}	\\
		&			
			&	3 or more
				&& \visible<3->{\key{ANOVA}}		\\
		\cline{2-5}
		&	\multirow{2}{*}{No}
			&	2
				&\multicolumn{2}{c}{\visible<4->{\key{Mann-Whitney-Wilcoxon}}%
					\footnote{\visible<5->{\aka
							\key{Mann-Whitney $U$ test} or
							\key{Wilcoxon rank sum} --
							Wilcoxon was a very busy person}}}	
\\
		&			
			&	3 or more
				&\multicolumn{2}{c}{\visible<6->{\key{Kruskal-Wallis}}}		\\
\hline
\multirow{2}{*}{\visible<7->{No}}
		& \multirow{2}{*}{\visible<7->{\textcolor{grey}{n/a}}}
			&	\visible<7->{2}
				&\multicolumn{2}{c}{\visible<9->{\key{Wilcoxon signed-rank}}} \\
		&			
			&	\visible<7->{3 or more}
				&\multicolumn{2}{c}{\visible<9->{\key{Friedman}}} \\
\hline
\hline
\\[-0.5em]
\visible<10->{Yes}
		&\visible<10->{Yes}
			&\visible<10->{Pairs}
				& \multicolumn{2}{c}{\visible<10->{\key{paired $t$-test}}}
\\[0.5em]
\hline
\hline

\end{tabular}
\end{center}
\end{frame}





\end{document}
